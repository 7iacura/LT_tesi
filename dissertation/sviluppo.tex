\chapter{Sviluppo}
\label{chp:sviluppo}

L'analisi dei requisiti ha evidenziato la necessità di un sito web in cui esporre e descrivere al cliente i servizi offerti, aggiornato e popolato da un componente del team di {\fem} senza per forza conoscenze e capacità tecniche informatiche; si é scelto quindi WordPress, un CMS molto diffuso.

Per la gestione degli ordini sia dal lato cliente che dal lato admin invece é stata progettata un soluzione su misura, utilizzando \emph{Django}, un web framework per lo sviluppo di applicazioni web. Lo sviluppo di essa é stato suddiviso in due parti in modo da distribuire il carico di lavoro, io mi sono occupato del lato front-end e due colleghi del lato back-end.

In questo capitolo verrà analizzata tutta la catena produdditvavadlgja da ... a ... .

\section{{\wp}}
\label{sec:wp}

{\wp} è una piattaforma software di content management system (CMS) ovvero un programma installato sul server che consente la creazione, gestione, distribuzione e manutenzione di un sito Internet \cite{wordpress}. É un progetto open-source creato da Matt Mullenweg e distribuito con la licenza GNU General Public License; é sviluppato in PHP con appoggio a MySQL come gestore di database.

{\wp} permette il download gratuito di tutti i suoi componenti dal sito \url{www.wordpress.org} per poterli installare sulla propria macchina. Esiste anche un servizio (a pagamento in base alle richieste) chiamato \emph{WordPress.com} che permette di costruire rapidamente il proprio sito web o blog basato su {\wp} senza la necessità di possedere un server o competenze tecniche specifiche.

\subsection{Caratteristiche di {\wp}}

{\wp} permette di estendere le proprie funzionalità con l'ausilio di opportuni plugin, ovvero moduli che aggiungono nuove caratteristiche ed elementi all'applicativo. I plugin possono essere gratuiti o a pagamento e possono fare molte fare di tutto, dal potenziare l'editor integrato di {\wp} all'inserire slideshow nelle pagine, e molto altro ancora. Come i plugin si possono trovare anche temi, estensioni che permettono di personalizzare l'aspetto del sito modificando sfondi, impaginazione, font, etc.

\subsection{{\wp} per {\fem}}

Per realizzare il \emph{Portale della Diagnostica Molecolare dedicato all'Avifauna}, dopo aver scelto il sottodominio \texttt{www.avifauna.fem2ambiente.com}, si é prima di tutto installato e configurato {\wp}.

Per farlo é stato necessario scaricare l'ultima versione dal sito \url{www.wordpress.org} (ad oggi, Ottobre 2015, l'ultima versione é la 4.3.1) e seguire le istruzioni nel file \texttt{readme}, in particolare:
\begin{itemize}
\item eseguire le opportune modifiche al file \texttt{wp-config.php} in un editor di testo;
\item creare un database dedicato utilizzando MySQL;
\item connettersi al server e caricare tutti i file relativi l'installazione di {\wp} nella cartella scelta (\texttt{/home});
\item configurare in modo appropriato visitando la pagina 

\texttt{http://avifauna.fem2mabiente.com/home/wp-admin/install.php}
\end{itemize}

Una volta terminata l'installazione si é potuto procedere con l'installazione degli appropriati plugin, temi e estensioni.

Per il tema la scelta é ricaduta su \emph{Everest} di YOOtheme (Versione: 1.0.11), 
