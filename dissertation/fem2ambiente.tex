\chapter{\fem}
\label{chp:fem}

In questo capitolo viene descritta l'azienda \textbf{\femsrl} in modo da contestualizzare i bisogni che hanno portato allo sviluppo della piattaforma web dedicata.

In particolare nelle sezioni \ref{sec:storia-fem} e \ref{sec:richiesta} sono illustrate prima nascita ed espansione della spin-off, poi bisogni, richieste e necessità legate alla piattaforma sviluppata.

\section{La storia di \fem}
\label{sec:storia-fem}

\begin{wrapfigure}{r}{0.32\textwidth}
  \begin{center}
    \includegraphics[width=0.3\textwidth]{images/logo-fem}
  \end{center}
\end{wrapfigure}

\textbf{\femsrl} è uno spin-off del Dipartimento di Biotecnologie e Bioscienze dell'{\unimib}, nato con l'intenzione di creare prodotti e servizi per il largo pubblico finalizzati alla conoscenza e tutela della biodiversità. La mission è supportare i consumatori nelle scelte, rendendoli consapevoli sulla qualità delle risorse ambientali, e in questo modo fornendo gli strumenti necessari a migliorare il loro stile di vita, tutelando l'ambiente \cite{fem2ambiente}.

Ad inizio 2007 é nato \textbf{\zpl} dall'incontro di Massimo Labra e Maurizio Casiraghi, due ricercatori del Dipartimento di Biotecnologie e Bioscienze dell'Università degli Studi di Milano Bicocca che si occupano rispettivamente di tematiche botaniche e zoologiche. Lo {\zpl} é un laboratorio di ricerca di zoologia, botanica e microbiologia che coniuga ricerca di base e applicata con progetti che prevedono l'utilizzo di un approccio molecolare \cite{zooplantlab}.

Nel gennaio 2010 grazie al contributo e i risultati della ricerca di {\zpl} viene fondato {\femsrl} dai quattro soci fondatori: Dott. De Mattia Fabrizio, Dott. Ferri Emanuele, Dott. Labra Massimo, Dott. Casiraghi Maurizio.

Food, Environment \& ManageMent (FEM2): alimentazione, ambiente e gestione razionale sono alcuni degli aspetti ai quali {\fem} dedica particolare attenzione, con l'ottica di fornire informazioni e strumenti per un utilizzo più consapevole delle risorse, in sintonia con il pianeta, utilizzando tecnologie e conoscenze derivanti dalla ricerca scientifica.

{\fem} dispone di moderni laboratori ospitati presso l’Università, nei quali vengono sviluppati e testati i nuovi prodotti, eseguite analisi su matrici ambientali (es: acqua, aria o alimenti), si svolgono analisi del DNA e vengono messe a punto metodiche innovative di caratterizzazione molecolare. Grazie ad essi oggi {\fem}, pur mantenendo le sue solide radici universitarie ed investendo nella ricerca, si è affermata anche come società commerciale e propone al mercato nazionale ed internazionale prodotti e servizi all'avanguardia nei settori dell'ambiente, del food e della diagnostica molecolare avanzata.

Negli ultimi anni è diventato leader di mercato nella \emph{diagnostica molecolare di avifauna} tramite PCR (analisi del DNA), ed é nata la necessita di sviluppare una piattaforma adatta per gestire tutte le fasi di analisi e vendita dei servizi.

\section{La richiesta}
\label{sec:richiesta}

Lo sviluppo di {\fem} sul mercato ha portato alla creazione del sito \url{www.fem2ambiente.com} con annesso un e-commerce che ad oggi permette la vendita di prodotti per l'analisi di acqua e aria, eco-prodotti per la casa


\begin{comment}

Il portale della diagnostica molecolare dedicato all'avifauna offre servizi di \emph{sessaggio}, \emph{diagnosi patologie} e \emph{Dna barcoding} disponibili per chi detiene, alleva o rivende avifauna.
Determinare il sesso precocemente e con sicurezza consente a chi intende assortire delle coppie di scongiurare perdite economiche e di tempo dovute all'assortimento di coppie formate da due maschi o due femmine, impossibili quindi a riproduzione. 
Il sessaggio da uovo consente a chi alleva a mano di soddisfare le richieste del cliente senza il rischio di fornire un soggetto del sesso non richiesto dopo un lungo lavoro di imprinting.

PATOLOGIE
I più recenti dati statistici indicano che la prevalenza di APV (poliomavirus) e BFDV (circovirus) in Italia sono rispettivamente del 31 e 26 %, valori che possono aumentare di oltre il doppio in determinati canali di vendita e scambi.

IDENTIFICAZIONE
L’innovativo servizio di identificazione tramite DNA (DNA barcoding) consente di stabilire la specie di appartenenza di qualsiasi soggetto a partire da diverse tipologie di materiale biologico (ad es. gusci di uovo, sangue, penne, piume, ecc.).

---

Per rendere più fruibile il servizio di analisi al cliente finale si é reso necessario creare un portale dedicato, il portale della \emph{Diagnostica Molecolare} che offre servizi di \emph{sessaggio}, \emph{diagnosi patologie} e \emph{Dna barcoding} disponibili per chi detiene, alleva o rivende avifauna.

La relazione cerca di descrivere lo sviluppo dalla base di un portale di questo tipo, e sarà composta da -N- parti principali: 

---
\end{comment}

