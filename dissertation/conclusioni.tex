\chapter{Conclusioni} 
\label{chp:conclusioni}
L’obiettivo dello stage é stato quello di sviluppare una piattaforma web per la creazione di un e-commerce personalizzato, facendo attenzione ad ogni parte del processo costruttivo.

La scelta del framework web é ricaduta su \emph{Django} grazie alla sua flessibilità e alle sue proprietà; é stato in grado supportare la realizzazione una struttura complessa, aiutando gli sviluppatori con alcuni tool inclusi per la costruzione del pannello admin, la traduzione per una piattaforma multilingua e la creazione di file pdf e file testuali per l'esportazione di dati.

Per lo sviluppo del sito durante il flusso di creazione e gestione dell'ordine dal lato cliente si sono sfruttate le più famose tecnologie per lo sviluppo web, basandosi sui linguaggi \emph{{\js}}, \emph{CSS}, \emph{HTML}, e un framework come \emph{Bootstrap}.

HTML alla base di tutte le pagine visitate connesso ai fogli di stile CSS per il sostegno alla struttura e al lato estetico; {\js} per le funzionalità aggiuntive e per la creazione di un interfaccia più usabile ed intutitiva per il cliente finale.

Bootstrap é stato un ottimo supporto per il livello più altro della costruzione delle pagine web grazie al suo tool completo di regole CSS e {\js}.

Per il alto divulgativo e di spiegazione delle funzionalità del sito la scelta é stata orientata su un CMS per fornire ai tecnici di {\fem} uno strumento facile da usare senza la necessità di conoscente tecniche in informatica, così si é deciso per l'installazione di {\wp}.

Il risultato é stato un sito fluido e funzionale che fonde una parte espositiva dei servizi forniti con la piattaforma di vendita vera e propria, nascondendo al cliente la complessità del sistema, ma contemporaneamente fornendo ai dipendenti di {\fem} uno strumento completo per la gestione di tutti i componenti, dal sito, al flusso degli ordini, alla gestione delle analisi fino alla generazione di attestati finali.