\chapter{Introduzione} 

\textbf{\femsrl} è uno spin-off del Dipartimento di Biotecnologie e Bioscienze dell'{\unimib}, nato con l'intenzione di supportare i consumatori nelle scelte dei prodotti e servizi biologici. Dopo la sua nascita nel 2010, si é sviluppando ed é cresciuto sempre più grazie al supporto fornito dall'Università per la ricerca e grazie al suo impatto sul mercato con prodotti innovativi e di facile uso.

{\fem} dispone di moderni laboratori ospitati presso l’Università, nei quali vengono sviluppati e testati i nuovi prodotti, e durante la crescita si sono concentrati anche su prodotti mirati all'avifauna. In un primo momento ha cominciato ad offrire servizi di analisi su uccelli e col passare del tempo il numero di analisi possibili é aumentato ed é ancora in crescita. É stato necessario quindi evolversi da un primo approccio col cliente attraverso fogli di excel e costruire una piattaforma in grado di gestire gli ordini e il flusso cliente.

Questo sviluppo é l'argomento dela seguente relazione ed é stato l'ambito del lavoro di stage.

É stato necessario l'utilizzo di diverse tecnologie in base alle richieste: dalla scelta di un CMS come \emph{{\wp}} per la parte del sito più informativa ed espositiva, in grado di essere facilmente usabile ad aggiornabile da persone non tecniche del settore informatico, alla scelta di framework per una piattaforma autonoma ma connessa al sito principale che permettessero la gestione completa di un flusso ordini personalizzato come quello del Portale Avifauna.
  
Nel capitolo~\ref{chp:fem} sarà descritta meglio l'azienda {\femsrl} e la sua storia, fino ad arrivare alla richiesta dello sviluppo della piattaforma web, nel capitolo~\ref{chp:wp} verrà descritta la scelta del CMS {\wp} e la sua installazione, mentre il capitolo~\ref{chp:sviluppo} é dedicato al vero e proprio sviluppo del Portale attraverso la descrizione di tutti i tool e framework utilizzati.

Per approfondire alcune porzioni di codice significative é stata scritta l'Appendice~\ref{app:codice} con attenzione ad alcuni particolari funzioni fornite dal framework Django.

Invece per descrivere meglio i compiti lato admin e il flusso cliente sono state scritte rispettivamente le appendici~\ref{app:admin} e \ref{app:cliente}.
