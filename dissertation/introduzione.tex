\chapter{Introduzione} 

\textbf{\femsrl} è uno spin-off del Dipartimento di Biotecnologie e Bioscienze dell'{\unimib} costituito nel 2010 con l'obiettivo di immettere nel mercato prodotti e servizi rivolti all'educazione, alla tutela delle risorse ambientali e alla conoscenza della biodiversità. 

{\fem} ha sede negli edifici universitari, un ambiente di lavoro ideale dove è possibile tradurre in applicazioni pratiche conoscenze che vanno dalle biotecnologie animali e vegetali sino alle scienze ambientali; l'incubazione universitaria fornisce inoltre l'accesso alle grandi strutture e alle strumentazioni più all'avanguardia che hanno permesso lo sviluppo di prodotti mirati all'avifauna. La quantità di servizi offerta è cresciuta, come la clientela, fino ad arrivare ad essere uno dei leader del mercato nel settore. È stato necessario quindi evolvere il primordiale approccio con il cliente attraverso fogli Excel, fino a costruire una piattaforma in grado di gestire il completo flusso di ordini.

Questo sviluppo è l'argomento della relazione ed è stato l'ambito del lavoro di stage.

È stato necessario l'utilizzo di diverse tecnologie in funzione delle richieste: dalla scelta di un CMS come \emph{{\wp}} per la parte del sito più informativa ed espositiva (in grado di essere facilmente usabile ad aggiornabile da persone non tecniche del settore informatico), alla scelta di un framework web per la creazione di una piattaforma autonoma che permettesse la gestione completa di un flusso ordini personalizzato come quello del Portale Avifauna.
  
Nel capitolo~\ref{chp:fem} è descritta la storia ed evoluzione dell'azienda {\femsrl} fino ad arrivare alla richiesta di sviluppo della piattaforma web. Nel capitolo~\ref{chp:wp} è descritta la scelta e l'installazione di {\wp} come Content Management System, mentre il capitolo~\ref{chp:sviluppo} è dedicato al vero e proprio sviluppo del Portale attraverso la descrizione di tutti i tool e framework utilizzati.

Per approfondire alcune porzioni di codice significative è possibile leggere l'Appendice~\ref{app:codice} che pone l'attenzione su alcuni particolari funzioni fornite dal framework Django.

Per descrivere meglio le azioni possibili sia lato admin che lato cliente sono state scritte rispettivamente le appendici~\ref{app:admin} e \ref{app:cliente}.
