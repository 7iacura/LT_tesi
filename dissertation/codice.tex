\chapter{Codice Sorgente}

In questa sezione sono riportate e commentate alcune parti del codice sorgente citate nel corso della discussione.

\section*{Utilizzo dei templates Django}
\label{app:templates}

Come illustrato nel corso della discussione Django fornisce uno strumento utile per centralizzare e non duplicare codice HTML, i \emph{templates}.

Si é costruito un file \texttt{base.html} (di seguito parte del codice semplificato) per includere in un file unico tutte le informazioni statiche e ripetute in tutti gli altri file HTML del progetto.

\begin{footnotesize}
\begin{verbatim}
<html>
<head>
  <title>fem2ambiente </title>
  <link rel="stylesheet" type="text/css" href="/static/css/bootstrap.css">
</head>

<body>
   <nav role="navigation">                        
     
   </nav>
  <main>
    
  </main>


  <script src="/static/js/bootstrap.js"></script>

</body>
</html>
\end{verbatim}
\end{footnotesize}
