\chapter{Codice Sorgente}
\label{app:codice}

In questa sezione sono riportate e commentate alcune parti del codice sorgente citate nel corso della relazione.

\section*{Templates Django}
\label{app:templates}

Come illustrato nel corso della discussione Django fornisce uno strumento utile per centralizzare e non duplicare codice HTML, i \emph{templates}.

Il file \texttt{base.html} (di seguito parte del codice semplificato) è stato necessario per includere in un unico documento tutte le informazioni statiche e ripetute in tutti gli altri file HTML del progetto.

\begin{footnotesize}
\begin{verbatim}
 <html>
 <head>
   <title>fem2ambiente </title>
   <link rel="stylesheet" type="text/css" href="/static/css/bootstrap.css">
 </head>
 <body>
    <nav role="navigation">                        
      
    </nav>   
   <main>
     
   </main>

   
     <script src="/static/js/bootstrap.js"></script>
   
 </body>
 </html>
\end{verbatim}
\end{footnotesize}

Si può notare che:
\begin{itemize}
 \item nell'intestazione del documento (\texttt{head}) il blocco \tag{title} contiene \dja{block title}, sintassi di Django che permette di includere all'interno del blocco un titolo diverso per ogni file HTML che viene esteso.
 \item \dja{include "login.html"} è la chiamata per includere in quella sezione della barra di navigazione il file dedicato al login, nel quale, in base allo stato del cliente (loggato o no), viene mostrato il form di login oppure i dati personali.
 \item \dja{block content} indica dove è inserito il codice HTML del file che estende.
 \item \dja{block jsexec} indica la sezione dedicata all'inserimento degli script {\js} in aggiunta o sovrascrivendo quelli già presenti nel file di base.
\end{itemize}

Di seguito un esempio di estensione del file \texttt{base.html}.

\begin{footnotesize}
\begin{verbatim}
 
 
 
   {{ block.super }} - Pagina Personale
 
 
 
   <div>[...]</div>
 
 
 
   {{ block.super }}
   <script type="text/javascript" src="/static/js/fledjed.js"></script>
 
\end{verbatim}
\end{footnotesize}

Si può notare che:
\begin{itemize}
 \item nella prima riga \dja{extends "base.html"} indica la necessità di prendere tutte le informazioni di intestazione e inclusioni di file statici CSS o {\js} dal file di base.
 \item in \dja{block title} (come in \dja{block jsexec}) è presente la riga di codice \{\{\texttt{block.super}\}\}; essa indica l'ereditarietà del contenuto del blocco titolo (o javascript) presente nel file di base ed esteso con il contenuto del file in questione; è un comportamento al \texttt{super} in Java. Il titolo risultante del codice appena descritto è \texttt{fem2ambiente - Pagina Personale}.
\end{itemize}


\section*{Traduzione}
\label{app:traduzione}
Django mette a disposizione un sistema per la gestione multilingua della piattaforma, per farlo è necessario inserire piccole parti di codice in tutti i punti che dovranno essere tradotti, dette \emph{translation strings}. Django provvede automaticamente in compilazione ad incapsulare le stringhe da tradurre e trasferirle in una serie di file appositi dove è inserita ogni stringa e il corrispondente spazio per la traduzione; una volta popolati i file (le traduzioni sono da inserire manualmente (Django non fornisce un traduttore incorporato) il sistema sarà pronto per supportare il multilingua.

Nel file \texttt{settings.py} vengono impostate le lingue in cui tradurre il sito con il seguente codice:
\begin{footnotesize}
\begin{verbatim}
 LANGUAGES = (
  ('it', gettext('Italian')),
  ('en', gettext('English')),
 )
\end{verbatim}
\end{footnotesize}

In ogni template HTML è poi necessario inserire in testa al file la riga di codice \dja{load i18n} per permettere a Django di individuare le stringe da tradurre in mezzo al contenuto. Esse inoltre sono rese meglio identificabili attraverso l'utilizzo di alcuni tag.

\begin{itemize}
 \item \texttt{trans} seguito dalla stringa da tradurre
  \begin{footnotesize}
  \begin{verbatim}
   
  \end{verbatim}
  \end{footnotesize}
 \item \texttt{trans} seguito dalla stringa da tradurre e la variabile ad essa associata; utile nel caso in cui la stringa viene ripetuta più volte all'interno del documento
  \begin{footnotesize}
  \begin{verbatim}
   
  \end{verbatim}
  \end{footnotesize}
 \item Un blocco identificato da \texttt{blocktrans} che include al suo interno la stringa da tradurre ed eventuali variabili definite in precedenza; esso è utile nel caso in cui si voglia lasciare all'interno porzioni in HTML, che però andranno ripetute in sede di traduzione
  \begin{footnotesize}
  \begin{verbatim}
    
     stringa da tradurre 
     nella quale si può inserire una o più variabili
     come {{esempio_variabile}}
   
  \end{verbatim}
  \end{footnotesize}
\end{itemize}

Procedimento analogo va applicato ai file {\js} che prevedono un output testuale che richiede traduzione; la stringa utilizzata per identificarlo è \texttt{gettext()}, utilizzata come nel seguente esempio:
\begin{footnotesize}
\begin{verbatim}
 document.write(gettext('stringa da tradurre'));
\end{verbatim}
\end{footnotesize}

Il risultato finale sono file generati da Django con l'estensione \texttt{.po} in cui viene indicato path, nome del file e riga di codice di ogni stringa da tradurre, e la stringa stessa inserita nel file originale seguita dallo spazio per inserire la traduzione.
\begin{footnotesize}
\begin{verbatim}
 #: path/del/file.py:23
 msgid "Benvenuti nel mio sito."
 msgstr "Welcome to my site."
\end{verbatim}
\end{footnotesize}

Per collezionare le stringhe dai tempaltes di tutto il progetto, compilarle e creare i file \texttt{.po}, da terminale si esegue
\begin{footnotesize}
\begin{verbatim}
 $ ./manage.py makemessages
 $ ./manage.py compilemessages
\end{verbatim}
\end{footnotesize}

Per collezionare le stringhe dei file {\js} invece da terminale si esegue, per ogni lingua
\begin{footnotesize}
\begin{verbatim}
 $ ./manage.py makemessages -d djangojs -l it
\end{verbatim}
\end{footnotesize}
facendo attenzione ad inserire correttamente la sigla corrispondente ad ogni lingua precedentemente definita nel file \texttt{settings.py}.