\documentclass[12pt,a4paper]{scrartcl}
\usepackage[utf8]{inputenc}
\usepackage[english]{babel}
%\usepackage{parskip}

% --personal definitions--
\def \fem {FEM2-Ambiente}
\def \femsrl {FEM2-Ambiente Srl}
\def \wp {WordPress}
\def \js {JavaScript}
\def \b {Bootstrap}

\author{
	Mattia Curatitoli \\ 
	matricola 735722 \\
	\\
	\textbf{Relatore}: Daniela Micucci \\
	\textbf{Correlatore}: Emanuele Ferri
}
\title{Development e-commerce avifauna.fem2ambiente.com}
\subtitle{
	Final report summary \\
 	October 2015
}
\date{}

\begin{document}
\maketitle

{\femsrl} is a spin-off of Department of Biotechnology and Biosciences of the University of Milan-Bicocca created in 2010 with the target to sell products and services related to education, to ambiental protection and biodiversity knowledge. 

{\fem} is situated in university buildings, an ideal place of work where is possibile to convert knowledges in pratical applications. University incubation can connect the company to the big structures and devices that permitted the progress of its products. Services offered has grown up, as the clients, until to become first leader of the avifauna’s market. So it was necessary to evolve the first approach with client through Excel files in order to build a platform that can handle the complexity of orders.

The argument of the relation and the scope of my stage is the development of the platform, in particular the target was to develop a platform to create a personalised e-commerce, paying attention to every step of the constructive process.

It was necessary to use different technologies: the explicative part of the website was built with {\wp}, a CMS (Content Management System) easy to use and to update by people not in the field of informatics.

A framework web was used to build the shopping and selling platform, that can support it autonomously and to handle a complex flow of orders personalised as that of Portale Avifauna.

In the first chapter is described the company’s history and evolution until to arrive to the request of web platform’s development.

In the second chapter is described the choice and installation of {\wp} as CMS, a software that installed on server give a platform simple to handle. This is a great solution to describe the company, its mission and services.

The third chapter is dedicated to the development of Portale Avifauna through description of tools and frameworks used.

Django was the choice of framework because of its flexibility and properties that could support a complex structure realisation, included a translate system and a file manager for templates and HTML files. Django can support dedicated tools to build the admin’s panel, to create pdf documents and textual files to the purpose of data and statistics exportation.

The web interface of clients and administrators was developed by using languages like {\js}, CSS, HTML and Bootstrap frameworks.

HTML was the base structure of all visited pages using the templates system supplied by Django and stylised by CSS files. {\js} was an useful solution for the additional functionalities of the web interface.

Bootstrap was a great support for the highest level of web interface thanks to its framework with CSS rules, {\js} and the font tools integration.

The result was a functional and flowing site that mixes an exposition part about the services supplied by {\fem} with a complete selling platform, hiding the complexity of the system to the customer and in the meantime giving to {\fem}’s employees a complete tool to follow the flow orders, to handle all site component and analysis execution.

There is an appendix to know more about interesting code portion related to Django framework.

Two other appendices are about all the possible action that respectively consumer and admin can do in the builded platform.

\end{document}