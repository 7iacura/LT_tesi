\chapter{\fem}
\label{chp:fem}
In questo capitolo è descritta l'azienda \textbf{\femsrl} in modo da contestualizzare i bisogni che hanno portato allo sviluppo della piattaforma web dedicata.

In particolare nelle sezioni~\ref{sec:storia-fem} e~\ref{sec:richiesta} sono illustrate prima nascita ed espansione della spin-off, poi bisogni, richieste e necessità legate alla piattaforma sviluppata.

\section{La storia di \fem}
\label{sec:storia-fem}

\begin{figure}
  \centering
  \includegraphics[width=0.3\textwidth]{images/logo-fem}
  \caption{logo}
  \label{fig:logo-fem}
\end{figure}

\textbf{\femsrl} è uno spin-off del Dipartimento di Biotecnologie e Bioscienze dell'{\unimib} nato con l'intenzione di creare prodotti e servizi per il largo pubblico finalizzati alla conoscenza e tutela della biodiversità. La mission è supportare i consumatori nelle scelte rendendoli consapevoli sulla qualità delle risorse ambientali, in modo da fornire gli strumenti necessari a migliorare il loro stile di vita, tutelando l'ambiente \cite{fem2ambiente}.

Ad inizio 2007 è nato \textbf{\zpl} dall'incontro di Massimo Labra e Maurizio Casiraghi, due ricercatori del Dipartimento di Biotecnologie e Bioscienze dell'Università degli Studi di Milano Bicocca che si occupano rispettivamente di tematiche botaniche e zoologiche. Lo {\zpl} è un laboratorio di ricerca di zoologia, botanica e microbiologia che coniuga ricerca di base e applicata con progetti che prevedono l'utilizzo di un approccio molecolare \cite{zooplantlab}.

Nel gennaio 2010 grazie al contributo e i risultati della ricerca di {\zpl} viene fondato {\femsrl} dai quattro soci fondatori: Fabrizio De Mattia, Emanuele Ferri, Massimo Labra e Maurizio Casiraghi.

Food, Environment \& ManageMent (FEM2): alimentazione, ambiente e gestione razionale sono alcuni degli aspetti ai quali {\fem} dedica particolare attenzione, con l'ottica di fornire informazioni e strumenti per un utilizzo più consapevole delle risorse, in sintonia con il pianeta, utilizzando tecnologie e conoscenze derivanti dalla ricerca scientifica.

{\fem} dispone di moderni laboratori ospitati presso l’Università, nei quali vengono sviluppati e testati i nuovi prodotti, eseguite analisi su matrici ambientali (es: acqua, aria o alimenti), si svolgono analisi del DNA e vengono messe a punto metodiche innovative di caratterizzazione molecolare. Grazie ad essi oggi {\fem}, pur mantenendo le sue solide radici universitarie ed investendo nella ricerca, si è affermata anche come società commerciale proponendo al mercato nazionale ed internazionale prodotti e servizi all'avanguardia nei settori dell'ambiente, del food e della diagnostica molecolare avanzata.

Negli ultimi anni è diventato leader di mercato nella \emph{diagnostica molecolare di avifauna} tramite PCR (analisi del DNA), ed è nata la necessita di sviluppare una piattaforma adatta per gestire tutte le fasi di analisi e vendita dei servizi.

\section{La richiesta}
\label{sec:richiesta}
La crescita di {\fem} sul mercato ha portato alla creazione del sito dedicato \url{www.fem2ambiente.com} basato su \emph{Joomla!}, un \emph{CMS} (Content Management System) molto diffuso, utile per la gestione dei contenuti del sito web senza la necessità di avere conoscenze tecniche \cite{joomla}. Ad esso è stata aggiunta l'estensione \emph{VirtueMart}, una soluzione open-source per la gestione dell'e-commerce che ad oggi permette l'acquisto di prodotti per l'analisi di acqua e aria rivolti a privati, imprese e per l'educazione nelle scuole, oltre ad eco-prodotti per la casa e per la cura degli animali \cite{virtuemart}.

Quando {\fem} ha cominciato ad offrire servizi di analisi su avifauna, l'utilizzo di fogli elettronici Excel è sembrata la soluzione migliore, ma col crescere della clientela e del numero di ordini è stato necessario pensare ad una alternativa. 

È nata così la richiesta di una piattaforma dedicata in grado supportare la vendita specifica dei servizi offerti di \emph{sessaggio}, \emph{diagnosi patologie} e \emph{Dna barcoding}, e il controllo del flusso ordini integrato con i procedimenti in laboratorio. Per fare ciò non é sufficiente l'utilizzo di un semplice tool e-commerce già esistente, di conseguenza sono stati sviluppati:
\begin{itemize}
\item il sito dedicato all'esposizione dei servizi offerti e delle informazioni riguardanti {\fem}: il \emph{Portale della Diagnostica Molecolare dedicato all'Avifauna} (capitolo~\ref{chp:wp})
\item la piattaforma lato server per il controllo completo del flusso ordini (capitolo~\ref{sec:server})
\item l'interfaccia per il flusso cliente durante la creazione e monitoraggio degli ordini, e un pannello admin per la gestione completa di ogni parte della piattaforma (capitolo~\ref{sec:client})
\end{itemize}

La mole di lavoro da compiere ha creato la necessità di dividere i compiti, così mi sono occupato dello sviluppo lato client e front-end dell'applicativo, oltre al controllo del CMS {\wp}; il lato server è stato invece compito di due colleghi.

Nei seguenti capitoli sono descritti ed analizzati tutti i passi fino alla messa online del sito con particolare attenzione al lato front-end. Nel dettaglio il capitolo~\ref{chp:wp} descrive le azioni per impostare correttamente il CMS, mentre il capitolo~\ref{chp:sviluppo} contiene la sezione~\ref{sec:server} che fornisce una panoramica sul lato server, mentre il lato client é descritto nella sezione~\ref{sec:client}.
